\documentclass[a4paper,11pt]{article}

\usepackage{latexsym}
\usepackage[empty]{fullpage}
\usepackage{titlesec}
\usepackage{marvosym}
\usepackage[usenames,dvipsnames]{color}
\usepackage{verbatim}
\usepackage{enumitem}
\usepackage[hidelinks]{hyperref}
\usepackage{fancyhdr}
\usepackage[english]{babel}
\usepackage{tabularx}
\usepackage{fontawesome5}
\usepackage{multicol}
\usepackage{graphicx}
\setlength{\multicolsep}{-3.0pt}
\setlength{\columnsep}{-1pt}
\input{glyphtounicode}

\pagestyle{fancy}
\fancyhf{}
\fancyfoot{}
\renewcommand{\headrulewidth}{0pt}
\renewcommand{\footrulewidth}{0pt}

% Adjust margins
\addtolength{\oddsidemargin}{-0.6in}
\addtolength{\evensidemargin}{-0.5in}
\addtolength{\textwidth}{1.19in}
\addtolength{\topmargin}{-.5in}
\addtolength{\textheight}{1.4in}

\urlstyle{same}

\raggedbottom
\raggedright
\setlength{\tabcolsep}{0in}

% Sections formatting
\titleformat{\section}{
  \vspace{-4pt}\scshape\raggedright\large\bfseries
}{}{0em}{}[\color{black}\titlerule \vspace{-5pt}]

% Ensure that generate pdf is machine readable/ATS parsable
\pdfgentounicode=1

%-------------------------
% Custom commands
\newcommand{\resumeItem}[1]{
  \item\small{
    {#1 \vspace{-2pt}}
  }
}

\newcommand{\resumeSubheading}[4]{
  \vspace{-2pt}\item
    \begin{tabular*}{1.0\textwidth}[t]{l@{\extracolsep{\fill}}r}
      \textbf{#1} & \textbf{\small #2} \\
      \textit{\small#3} & \textit{\small #4} \\
    \end{tabular*}\vspace{-7pt}
}

\newcommand{\resumeProjectHeading}[2]{
    \item
    \begin{tabular*}{1.001\textwidth}{l@{\extracolsep{\fill}}r}
      \small#1 & \textbf{\small #2}\\
    \end{tabular*}\vspace{-7pt}
}

\newcommand{\resumeSubHeadingListStart}{\begin{itemize}[leftmargin=0.0in, label={}]}
\newcommand{\resumeSubHeadingListEnd}{\end{itemize}}
\newcommand{\resumeItemListStart}{\begin{itemize}}
\newcommand{\resumeItemListEnd}{\end{itemize}\vspace{-5pt}}

%-------------------------------------------
%%%%%%  RESUME STARTS HERE  %%%%%%%%%%%%%%%%%%%%%%%%%%%%

\begin{document}

%----------HEADING----------
\begin{center}
    {\Huge \scshape Kevin Lau} \\ \vspace{3pt}
    \small \raisebox{-0.1\height}\faMapMarker\ Cambridge ~
    \href{tel:+447311889669}{\raisebox{-0.2\height}\faPhone\  \underline{+44 7311 889669}} ~
    \href{mailto:kevinlauofficial01@gmail.com}{\raisebox{-0.2\height}\faEnvelope\  \underline{kevinlauofficial01@gmail.com}} \\
    \vspace{1pt}
    \href{https://linkedin.com/in/kevinlau01}{\raisebox{-0.2\height}\faLinkedin\ \underline{linkedin.com/in/kevinlau01}} ~
    \href{https://github.com/booth-algo}{\raisebox{-0.2\height}\faGithub\ \underline{github.com/booth-algo}}
    \vspace{-8pt}
\end{center}

%-----------EDUCATION-----------
\section{Education}
\resumeSubHeadingListStart
\resumeSubheading
{Imperial College London}{Oct 2022 -- Jun 2026}
{MEng Electronic and Information Engineering (Computer Engineering) - \emph{Predicted 1st class honours}}{}
\resumeSubHeadingListEnd

%-----------WORK EXPERIENCE-----------
\section{Work Experience}
\resumeSubHeadingListStart

\resumeSubheading
{DeepWok Lab (Imperial x Cambridge) \href{https://deepwok.github.io/}{\scalebox{0.75}\faLink}}{Jul 2024 -- ongoing}
{3D Gaussian Splatting Quantisation and Acceleration Hardware}{}
\resumeItemListStart
\resumeItem{Developing custom quantised hardware for 3D Gaussian Splatting in SystemVerilog with custom cocotb testbenches}
\resumeItem{Implemented quantisation-aware training for 3DGS using PyTorch, achieving similar PSNR benchmarks to the official CUDA implementation}
\resumeItem{Applied in-house compiler MASE's custom quantisers to evaluate the best quantisation scheme for hardware design}
\resumeItemListEnd

\resumeSubheading
{Imperial College London}{Jul -- Sept 2024}
{University Course FPGA Module Design}{}
\resumeItemListStart
\resumeItem{Redesigned the 2nd year Information Processing module teaching content and lab practicals from scratch}
\resumeItem{Introduced concepts of hardware-software codesign and embedded development with Verilog, C++ and Python}
\resumeItem{Emphasized on practical skills development with the Xilinx toolchain and edge-computing applications}
\resumeItemListEnd

\resumeSubheading
{Imperial College London}{Oct 2023 -- Mar 2024}
{Undergraduate Teaching Assistant}{}
\resumeItemListStart
\resumeItem{Taught Year 1 students in the EEE department for the Programming for Engineers module}
\resumeItem{Guided students on learning fundamental C++ concepts and developing object-oriented programming skills}
\resumeItemListEnd

\resumeSubheading
{DiTa Limousine Limited \href{https://ditaexpress.com/}{\scalebox{0.75}\faLink}}{Jul -- Sept 2023}
{Full-stack Web Developer}{}
\resumeItemListStart
\resumeItem{Developed a responsive and interactive website using ReactJS and Framer Motion for the company website, which enhanced user engagement and contributed to a 50\% increase in new limousine service bookings}
\resumeItem{Hosted the website on a self-managed Ubuntu Virtual Private Server, gaining experience with the Linux shell and server management using NGINX}
\resumeItemListEnd

\resumeSubHeadingListEnd

%-----------PROJECTS-----------
\section{Projects}
\resumeSubHeadingListStart

\resumeProjectHeading
{\textbf{Graphics Processing Unit (TauriGPU)} \href{https://github.com/AyuubMohamud/TauriGPU}{\scalebox{0.75}\faLink} $|$ \emph{SystemVerilog, Python, GLSL}}{Jul 2024 -- ongoing}
\resumeItemListStart
\resumeItem{Developed an open-source programmable GPU compatible with OpenGL ES2 and Xilinx FPGAs}
\resumeItem{In progress of building an LLVM backend for TauriGPU's ISA to enable GLSL compilation}
\resumeItemListEnd

\resumeProjectHeading
{\textbf{Autonomous Balance Bot with Incident Management Platform} \href{https://github.com/TorturedEngineersDept/BalanceBot}{\scalebox{0.75}\faLink} $|$ \emph{Python, ROS 2}}{May -- June 2024}
\resumeItemListStart
\resumeItem{Led development of the autonomous navigation and physical incident detection system using SLAM and ROS 2}
\resumeItem{Developed a Frontier-based exploration algorithm to enable autonomous exploration capabilities in completely unknown dynamic environments}
\resumeItem{Physically implemented system on a Raspberry Pi 4 with a 2D LiDAR sensor and a camera}
\resumeItemListEnd

\resumeProjectHeading
{\textbf{C90 to RISC-V Compiler} \href{https://github.com/saturn691/ReaverCompiler}{\scalebox{0.75}\faLink} $|$ \emph{C++, RISC-V Assembly}}{Dec -- Mar 2024}
\resumeItemListStart
\resumeItem{Developed a compiler with advanced features, e.g. N-dimensional array support and efficient memory management}
\resumeItem{Placed 1st out of 48 teams, achieving 90\% pass rate in seen and unseen test cases}
\resumeItemListEnd

\resumeProjectHeading
{\textbf{FPGA Computer Vision Acceleration for ESP32 WiFi Car Racing System} $|$ \emph{Xilinx, C++}}{Feb -- Mar 2024}
\resumeItemListStart
\resumeItem{Built a commercializable hardware racing game with AWS cloud backend and implemented powerups using OpenCV}
\resumeItem{Developed hardware IPs for local OpenCV acceleration on the PYNQ-Z1 FPGA using the Xilinx toolchain}
\resumeItemListEnd

\resumeProjectHeading
{\textbf{Software-Hardware Low Latency Algorithmic Trading System with FPGA} $|$ \emph{Xilinx, Python}}{Feb 2024}
\resumeItemListStart
\resumeItem{Utilised FPGA to accelerate moving average indicators to identify market opening convergence opportunities using the Xilinx toolchain}
\resumeItem{Top 5 finalist at IC Hack 24's Optiver trading challenge out of 20+ teams, invited to present trading strategy to Optiver representatives}
\resumeItemListEnd

\resumeProjectHeading
{\textbf{RISC-V CPU} \href{https://github.com/booth-algo/RISC-V-T24}{\scalebox{0.75}\faLink} $|$ \emph{SystemVerilog, C++, RISC-V Assembly}}{Nov -- Dec 2023}
\resumeItemListStart
\resumeItem{Developed a single-cycle RISC-V 32I processor that runs all base instructions using SystemVerilog}
\resumeItem{Implemented pipelining and direct-mapped cache to improve processing and memory access speed}
\resumeItem{Placed 1st out of 24 teams in both quality of verification and codebase documentation}
\resumeItemListEnd

\resumeSubHeadingListEnd

%-----------SKILLS-----------
\section{Technical Skills}
\begin{itemize}[leftmargin=0.15in, label={}]
    \small{\item{
    \textbf{Programming Languages}{: C, C++, Python, JavaScript, RISC-V Assembly} \\
    \textbf{Hardware Description Languages}{: SystemVerilog, Verilog, VHDL} \\
    \textbf{Toolchain}{: CUDA, OpenGL, PyTorch, Verilator, cocotb, ROS 2, NumPy, OpenCV, ReactJS, NGINX, Git, Conda} \\
    \textbf{FPGA toolchain}{: Vivado, Vitis, Quartus Prime} \\
    \textbf{Languages}{: English (native), Cantonese (native), Mandarin Chinese (working proficiency)}
    }}
\end{itemize}

\end{document}